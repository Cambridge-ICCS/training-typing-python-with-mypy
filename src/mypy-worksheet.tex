\documentclass{article}

\usepackage{minted}
\usepackage{url}
\usepackage{enumitem}
\usepackage[margin=2cm,top=0.75cm]{geometry}

\title{Using types to rule out bugs: mypy worksheet}
\author{Dominic Orchard}
\date{}

\begin{document}
\maketitle

\noindent
This short worksheet will take you through a few exercises
to practice using mypy. Here is an example as a reminder
of the syntax for inserting type specifications:

\begin{minted}{python}
def myAbs(x : float) -> float:
  """Take the absolute of the floating-point input"""
  if x < 0:
    return (-x)
  else:
    return x
\end{minted}
%
You can find the standard documentation for Python's
\texttt{typing} library at:

\begin{center}
  \url{https://docs.python.org/3/library/typing.html}
\end{center}

\begin{enumerate}
\item Define a function \texttt{mean} to take the mean of two
  floats, giving its full type specification.

  Check that \texttt{mypy} accepts this (i.e., run \texttt{mypy yourfile.py}).

\item Experiment with modifying your function to something that
 would produce a runtime type error and check how \texttt{mypy} reports this.

\item Using \texttt{reveal\_type} find out what the type of
  the \texttt{len} function is according to \texttt{mypy}.

  (You might like to guard this with a conditional checking the
  \texttt{TYPE\_CHECKING} value so you can leave it in and
  run your code later).
\end{enumerate}
%
Consider the following Python code to calculate
the mean of a list:
%
\begin{minted}{python}
def meanN(xs : list[float]) -> float:
  sum = 0.0
  for x in xs:
    sum+=x
  return (sum / len(xs))
\end{minted}
%
There is nothing specific in this code to lists;
it can be generalised to arbitrary iterable data types.
Note however, that we use \texttt{len} which requires a
\texttt{Sized} value, and iteration requires \texttt{Iterable}.
Instead we can use the \texttt{Collection} class from the \texttt{typing} module
  (which inherits from both \texttt{Iterable} and \texttt{Sized}).

\begin{enumerate}[resume]
\item Import the \texttt{Collection} class and modify the above to work on arbitrary
  collections.

\item Define a generalisation of
  \texttt{meanN} that takes an additional function parameter.
  Instead of computing the mean directly, this new function
  \texttt{meanGen} should
  sum up all the elements of the collection then apply
  the parameter function to this value to produce the output.

  Give it a type signature with
  the most general type.

  \textit{Hint: Use the \texttt{Callable} class.
    You can introduce a parametric type variable
    via: } \texttt{T = TypeVar('T')}

  Write some code to use this to compute the mean of some test data.

\item (\textbf{Challenge}) We can further generalise \texttt{meanGen} into a
  function \texttt{reducer} which captures the general pattern
  of reducing a collection into a single value, taking as inputs:
  %
  \begin{itemize}
  \item A collection of some arbitrary element type \texttt{T};
  \item A function with two inputs for combining \texttt{T} values with
    a partially reduced value \texttt{S}, to produce a new reduced value \texttt{S};
  \item An initial reduced value \texttt{S};
  \item A final transformation function mapping from the reduced value \texttt{S} to
    the final result of type \texttt{U}.
  \end{itemize}
  %
  Implement this general reducer and redefine the mean of a collection
  in terms of it, giving a simple test.
\end{enumerate}
%
Solutions: \url{https://github.com/Cambridge-ICCS/training-typing-python-with-mypy/blob/main/mypy-worksheet-solutions.py}

\end{document}
